\section{Future work}
Although the implementation seems to generally work, there are some areas,
which I simply have not had the time to improve upon. These parts are
referenced here as what could be worked on further.

\subsection{Optimizations}
By no means is this implementation optimal. The general goal of the project is
to keep it as easy to understand as possible, and this has a large effect on
the overall optimizations. There are multiple cases of this, but I will just
mention what I think would lead to the best performance gain:

\begin{itemize}
  \item Making it such that we don't store the state of every single node. The
    current implementation stores the state for every node in the CFG. This is
    highly inefficient. If a node is visited only once, we could pass its state
    directly to the next node, eliminating the need to store the state of the
    incoming node.. We would most likely have to do one pass through of the CFG
    to check for such nodes, but afterwards we would save a lot of the memory
    usage that probably occur on larger programs.
  \item Narrowing and Widening are currently implemented by checking every
    single register and memory cell to see if the cell/register has changed
    since the last state. This is highly inefficient, and it should be possible
    to reduce the number of checked elements. For example for an instruction
    which sends the value to a register, then we would only have to check that
    single register. The reason this was not implemented is because of instructions such as:
    \begin{verbatim}
      stb [r0], 10
    \end{verbatim}
    Here we are writing to the memory location by the value of r0. r0 is seen
    as an interval, so in this case we would have to handle the interval as the
    possible memory cells that have to be updated instead of a single index.
    For the sake of time, this was not implemented
\end{itemize}


\subsection{Unhandled actions}
There are two main areas where I have taken a shortcut and not handled the instructions. Those two are explained here.

\paragraph{Binary operators: } The binary operators are not evaluated, instead
I simply return the top interval whenever we reach a binary operator. Binary
operators would of course give us some further information about the current
interval, but due to time constraints these were not implemented.

\paragraph{Conditional jumps: } 
In the provided snippet is how we handle the conditional jumps.
\begin{lstlisting}[language={haskell}, numbers=none]
case jmp of
  Jeq -> equalInterval i1 i2
  Jne -> notEqualInterval i1 i2
  Jlt -> lessThanInterval i1 i2
  Jle -> lessThanEqualInterval i1 i2
  Jgt -> greaterThanInterval i1 i2
  Jge -> greaterThanEqualInterval i1 i2
  Jslt -> lessThanInterval i1 i2
  Jsle -> lessThanEqualInterval i1 i2
  Jsgt -> greaterThanInterval i1 i2
  Jsge -> greaterThanEqualInterval i1 i2
  _ -> Value (i1, i2) -- For other jumps, we don't refine
\end{lstlisting}

Here we can see that signed and unsigned usage is not taken into account, and
the Jset instruction is also not handled. The Jset instruction is simply not
implemented, and we just return our input and with the signed and unsigned one
could argue, that we should be more restrictive on the interval. If we ever
reach an unsigned instruction, then we would know that the register could never
be negative, instead of just using the default interval calculations.
