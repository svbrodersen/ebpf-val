\section{Implementation}

\subsection{Definitions}\label{sec:def}
\begin{lstlisting}[language={haskell}, caption={Bound data type}, label={lst:def}]
data Bound = NegInf | Val Integer | PosInf deriving (Eq, Ord)
data Interval = Interval Bound Bound deriving (Eq, Ord)

// Bottom M
data BottomM a = Bottom | Value a
  deriving (Eq, Ord)

// IntervalM
type IntervalM = BottomM Interval

type Mem = Array Int IntervalM

data State = State
  { registers :: Array Int IntervalM
  , memory :: Mem
  }
  deriving (Eq)
\end{lstlisting}
As mentioned I make use of the infinite interval. This is of course not
something machines can correctly represent, and as such I use the Bound data
type, whose definition can be seen in Listing~\ref{lst:def}. Bound can thus
take up 3 different "values". Either it is NegInf($-\infty$), some value or
PosInf($\infty$). Note, that because we define it in the order we have, then
NegInf will be seen as smaller than any Val and PosInf will be seen as larger
than any Val. This allows us to simply use the default min and max functions on
the Bound type.

Along with Bound there is also the definition of the Interval type, which
contains two Bound values, the first indicating the lower bound and the second
indicating the upper bound.

This now allows me to represent all the interval values except for $\bot$. For
$\bot$ instead of simply adding the value to the Interval type, I decided to
make use of BottomM. BottomM is defined exactly the same way as the Maybe monad
usually used in Haskell. The only difference between the two is that BottomM
makes use of Bottom instead of Nothing, and Value x instead of Just x.

This choice was made as the large majority of operations on both intervals and
bounds return Bottom, if any of the inputs are Bottom. This is the exact same
functionality from the Maybe monad, and saves a lot of repetitive code checking
the specific cases.

At last, we have the definition of the State. The state simply contains two
arrays. One which maps the registers to some IntervalM and then the same for
each of the memory cells. A state is then created for each of the nodes in the
control flow graph when initializing the setup.

\subsection{Operator implementation}\label{sec:opImp}
The definitions from Section~\ref{sec:def} allows me to define functions as the type signature:
\begin{lstlisting}[language={haskell}]
binaryBound :: Bound -> Bound -> BottomM Bound
unaryBound :: Bound -> BottomM Bound

// Interval operators
type IntervalM = BottomM Interval
binaryInterval ::  Interval -> Interval -> IntervalM
unaryInterval ::  Interval -> Interval -> IntervalM

// Interval comparisons
binaryInterval :: Interval -> Interval -> 
                  BottomM (Interval, Interval)
unaryInterval ::  Interval -> 
                  BottomM (Interval, Interval)
\end{lstlisting}

\subsubsection*{Examples}
\begin{lstlisting}[language={haskell}, caption={Full division implementation}, label={lst:div}]
divBound :: Bound -> Bound -> BottomM Bound
divBound (Val x) (Val y)
  | y == 0 = Bottom
  | otherwise = Value $ Val (x `div` y)
divBound _ PosInf = Value $ Val 0
divBound _ NegInf = Value $ Val 0
divBound NegInf (Val y)
  | y > 0 = Value NegInf
  | y < 0 = Value PosInf
  | y == 0 = Bottom
divBound PosInf (Val y)
  | y > 0 = Value PosInf
  | y < 0 = Value NegInf
  | y == 0 = Bottom
divBound _ _ = Bottom

divInterval :: Interval -> Interval -> IntervalM
divInterval ab@(Interval a b) cd@(Interval c d)
  | Val 1 <= c = do
      ac <- divBound a c
      ad <- divBound a d
      bc <- divBound b c
      bd <- divBound b d
      Value $ Interval (min ac ad) (max bc bd)
  | d <= Val (-1) = do
      bc <- divBound b c
      bd <- divBound b d
      ac <- divBound a c
      ad <- divBound a d
      Value $ Interval (min bc bd) (max ac ad)
  | otherwise =
      let first = do
            t1 <- intersectInterval cd (Interval (Val 1) PosInf)
            divInterval ab t1
          second = do
            t2 <- intersectInterval 
                    cd
                    (Interval NegInf (Val (-1)))
            divInterval ab t2
       in unionIntervalM first second
\end{lstlisting}  
\paragraph{Division:}
As an example of division of intervals are provided in Listing~\ref{lst:div}.
divBound follows our general intuition, where the main difference is that
division by 0 is not undefined, but rather just returns bottom. 

The implementation of divInterval is more complex and requires special
handling. It first makes sure that c is greater than or equal to 1. That way
we are not dividing by 0, which would result in bottom, which we do not want
for the entire interval. If the check for c fails, then we instead attempt to
evaluate if d is less than or equal to -1. This would be the case, if the
entire interval is negative. If both of these evaluations fails, then we know
that we do not have an entirely positive nor entirely negative interval, and as
such we go to the recursive call. 

In the recursive call, we simply find the intersection between all possible
positive numbers and our divisor, and attempt to divide ab with it and then the
same for the negative values. At last, we just take the union between the two
intervals.

In this case, there is never the possibility of dividing by zero, but using the
bottomM monad, we don't have to check for this case explicitly.


\begin{lstlisting}[language={haskell}, caption={Union interval implementation}, label={lst:union}]
unionIntervalM :: IntervalM -> IntervalM -> IntervalM
unionIntervalM m1 m2 =
  case (m1, m2) of
    (Bottom, Value i2) -> Value i2
    (Value i1, Bottom) -> Value i1
    (Value i1, Value i2) -> unionInterval i1 i2
    (_, _) -> Bottom
\end{lstlisting}
\paragraph{Union Interval:} The union interval is the one operator for which Bottom is the neutral element.
This means, that if one of the inputs is Bottom, then we should not just return
Bottom ourselves. As such, the implementation of union can be seen in
Listing~\ref{lst:union}, where if only one input is bottom, then we simply
return the other input.


\subsection{Interpretation}
Using eBPF-tools, we are given a control flow graph, which is represented as:
\begin{lstlisting}[language={haskell}, numbers=none]
  Set (Label, Trans, Label)
\end{lstlisting}
This means, that we are given a set of tuples, where the first element is the
source of the transformation, followed by the transformation and at last the
node we end in. 
\begin{lstlisting}[language={haskell}, caption={The handleTrans functnion}, label={lst:handleTrans}]
handleTrans :: State -> Trans -> State
handleTrans state Unconditional = state
handleTrans state (NonCF inst) = handleNonCF state inst
handleTrans state (Assert jmp (Reg lhs)) regimm = ...
\end{lstlisting}
The handling of Trans is separated into a separate function,
\texttt{handleTrans}, which can be seen in Listing~\ref{lst:handleTrans}. The
function works by getting an initial State(state), and then taking the
transformation into account, before returning a new updated State. For example,
for the unconditional jump, we simply return the state we got as input, as we
gain no new information from the transformation. In the following sections, I
will go into more details with an example of the NonCF(non-Control flow) and
the Assert (conditional jump).


\subsubsection{Non-Control flow}
\begin{lstlisting}[language={haskell}, caption={handleNonCF}, label={lst:handleNonCF}, escapechar=@]
handleNonCF :: State -> Instruction -> State
handleNonCF state inst =
  case inst of
    Binary _ alu reg regimm -> handleBinary state (alu, reg, regimm)
    Unary _ alu reg -> handleUnary state (alu, reg)
    Store _ (Reg dst) offset regimm ->
      storeMemory dst offset regimm
    Load _ (Reg dst) (Reg src) offset ->
      loadMemory dst offset (R (Reg src))
    LoadImm (Reg dst) imm ->
      storeMemory dst Nothing (Imm imm)
    _ -> state
 where
  loadMemory :: Int -> Maybe MemoryOffset -> RegImm -> State
  loadMemory dst off (R (Reg src)) =
    let intv = registers state Array.! src @\label{line:intv}@
     in case getBounds intv off of
          Nothing -> state
          Just (loSrc, hiSrc) ->
            -- We calculate the new interval as the union between all possible memory locations
            let newInterval = Prelude.foldl unionIntervalM Bottom $ [memory state Array.! i | i <- [loSrc .. hiSrc]] @\label{line:newInterval}@
             in state{registers = registers state // [(dst, newInterval)]}
  loadMemory dst _ (Imm imm) =
    -- offset not used when loading immediate value
    let newInterval = fromInteger $ fromIntegral imm
     in state{registers = registers state // [(dst, newInterval)]}
  ...
  getBounds :: IntervalM -> Maybe MemoryOffset -> Maybe (Int, Int)
  getBounds (Value (Interval lo' hi')) off' =
    -- Make sure we have some memory we can index
    let off = maybe 0 fromIntegral off'
     in let lo = case max lo' (Val (-off)) of
              -- Lower bound has to be the maximum of -off and our item. We then add off after
              Val x' -> x' + off
              _ -> 0
         in let hi = case min hi' (Val (511 - off)) of
                  -- Upper bound has to be the maximum of 511-off and our item. We then add off after
                  Val x' -> x' + off
                  _ -> 511
             in Just (fromInteger lo, fromInteger hi)
  getBounds Bottom _ = Nothing
\end{lstlisting}


When we are given a non-control flow action, then we gain information about the
input state. The Binary and Unary calculations just make use of the interval
calculations directly that we have seen from Section~\ref{sec:opImp}.

I have decided to include the implementation of the loadMemory in
Listing~\ref{lst:handleNonCF}. This function is called, when we have an eBPF
instruction such as:
\begin{lstlisting}[language={haskell}, numbers=none]
ldxb r5, [r0]
\end{lstlisting}
Intuitively this instruction wants us to get the value x in r0, then read
memory location x and save the result in register r5. However, as mentioned in
Section~\ref{sec:introductionMem}, x is not a single value in our analysis and
rather it is an interval. The way we handle this is by first gathering the
interval value of r0 on line~\ref{line:intv}. Then, we find the valid boundary,
i.e., we only want to take the value of the interval, which lies within our
memory region. If we currently assume that r0 is $\top$, then we only want to look
in the memory of [0, 511], as anything else would be out of bounds. The bounds
function then also takes into account if we want to offset the value, in which
case we simply subtract the value from the interval.

Once we have the valid memory interval, then on line~\ref{line:newInterval} we
calculate the union of all these memory cells, which gives us a new single
interval for the possible values r5 could take on after the instruction. 

The store memory just works in the opposite direction. Here we load the
immediate value or the value in a register, and then we offset when writing to
the interval of possible memory locations.


\subsection{Work-set algorithm}\label{sec:workset}
\begin{lstlisting}[language={haskell}, caption={Workset algorithm}, label={lst:workset}, escapechar=@]
workSetAlgorithm :: CFG -> Map Label State -> CFG -> Map Label Int -> Map Label State
workSetAlgorithm graph states worklist counters
  | Set.null worklist = states @\label{line:base}@
  | otherwise =
      let (l, instr, n) = Set.elemAt 0 worklist
          w' = Set.deleteAt 0 worklist
          current_state = getPreviousState l
          evalState = handleTrans current_state instr
          oldState = states Map.! n
          newState = unionState evalState oldState
        in if oldState == newState @\label{line:stateCheck}@
            -- We don't add anything else
            then workSetAlgorithm graph states w' counters
            else
              workSet' w' n newState oldState
 where
  getSuccessors n' = Set.filter (\(l', _, _) -> l' == n') graph
  getPreviousState l = states Map.! l
  workSet' w' n newState oldState =
    -- Check the counter to figure out if we widen or not
    let total_count = counters Map.! n
        newWorkset = Set.union w' successors
        newStates = Map.insert n newState states
        successors = getSuccessors n
        res
          -- Regular work set algorithm, if we have reached the node less than 4 times
          | (total_count < wideningCount) =
              let newCount = Map.insert n (total_count + 1) counters
               in workSetAlgorithm graph newStates newWorkset newCount
          -- Widening if we have been here 4 times.
          | (total_count == wideningCount) =
              let newStates' = Map.insert n (widenState oldState newState) states
                  newCount = Map.insert n (total_count + 1) counters
               in workSetAlgorithm graph newStates' newWorkset newCount
          -- Narrow state until we have reached the node 10 times
          | (wideningCount < total_count) && (total_count < narrowingCount) =
              let newStates' = Map.insert n (narrowingState oldState newState) states
                  newCount = Map.insert n (total_count + 1) counters
               in workSetAlgorithm graph newStates' newWorkset newCount
          | otherwise = states
     in res
\end{lstlisting}
This leads us to the work-set algorithm itself. The workSetAlgorithm function
implements just this. The function takes in the full control-flow graph(graph),
states (which maps a node(Label) to some State), the workset(a new CFG which
contains the tuples (Label, Instr, Label) that remain in the work-set) and
counters (a Map from node(Label) to counters(Int). 

First, on line~\ref{line:base} we check the base case. If the current workset
is empty, then we just return the states.

Otherwise, we remove the first element from the workset, get the state of the
label for which we came from, and then evaluate the instruction with
handleTrans from before. handleTrans then gives us a new evaluated state given,
which in Equation~\ref{eq:1}(From Section~\ref{sec:introduction}), this would
correspond to $(\llbracket exp \rrbracket s)$. We then have to take the union
between this new evalState and the previous state from the current node. This
way, if a previous transition(Label, Instr, Label) has already reached this
node, then we correctly take the union between the two.

Line~\ref{line:stateCheck} then checks if this transition lead to a state
different than the previous one. If we have the same state, then we add nothing
to the work-set, and simply call our workSetAlgorithm again just with the
currently checked transition.

If instead we have a changed state, then we go to the workSet' function. The function works by:
\begin{enumerate}
  \item Get the total number of times we have reached the current Node(total\_count).
  \item Get all transitions, which start with the current node n(successors).
  \item Create the new work-set with all transitions that come after the current one(newWorkset).
  \item Create the new states, where we simply substitute the current nodes state with the newState(newStates).
  \item At last, we check how many times we have reached the current node. And do one of the following
    \begin{itemize}
      \item If we have reached the node less than wideningCount times, then we just call the worksetAlgorithm again with the new values, and increment the counter of the current node
      \item If we have reached the current node wideningCount amount of times, then we do a single windening step, following the definition from Equation~\ref{eq:widening} before the recursive call.
      \item If we have reached the current node more than wideningCount times, but less than the narrowingCount times, then we do narrowing following the definition of Equation~\ref{eq:narrowing} before the recursive call.
      \item If we have reached the current node more than narrowingCount times, then we decide that we probably will not get a better estimate, and simply return the states stopping the recursive calls.
    \end{itemize}
\end{enumerate}

\subsubsection{Calling the worksetAlgorithm}
\begin{lstlisting}[language={haskell}, caption={Calling the workset algorithm}, label={lst:callworkset}]
let graph = cfg prog
    labels = Set.toList $ allLabels graph
    initial_states = Map.fromList [(l, bottomState) | l <- labels]
    initial_counters = Map.fromList [(l, 0) | l <- labels]
    final_states = workSetAlgorithm graph initial_states (Set.singleton $ Set.elemAt 0 graph) initial_counters
    (_, exit) = Map.findMax final_states
    output = printf (printState exit)
\end{lstlisting}

The workSetAlgorithm can then be called as shown in
Listing~\ref{lst:callworkset}. First, we generate the empty states for each of
the nodes, then we initialize all counters(that keep track of the number of
times we have visited a node) to zero, before running the function. The
function then gives us back a new Map of labels to States, and we simply return
the largest Label as our final State, as this is when we exit the program.
However, the final\_states variable contain the final states for all of the
nodes within the graph.
