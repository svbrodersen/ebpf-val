\documentclass{article}
\usepackage{amsmath}
\usepackage{amsfonts}
\usepackage{amssymb}
\usepackage{graphicx}
\usepackage{listings}
\usepackage{stmaryrd}

\title{Value Analysis for eBPF}
\author{Simon}
\date{\today}
\lstset{
    basicstyle=\ttfamily\footnotesize,
    breakatwhitespace=false,         
    breaklines=true,                 
    captionpos=b,                    
    keepspaces=true,                 
    numbers=left,                    
    numbersep=5pt,                  
    showspaces=false,                
    showstringspaces=false,
    showtabs=false,                  
    tabsize=2
}

\begin{document}

\maketitle

\begin{abstract}
This report presents a value analysis for a subset of the eBPF language, called
micro-eBPF. The analysis is based on the theory of abstract interpretation
using intervals to approximate the set of possible values for registers and
  memory at each program state.
\end{abstract}

\tableofcontents

\section{Introduction}
The goal of this project is to implement a value analysis for a subset of the
eBPF language. This analysis aims to compute an approximation of the values of
registers and memory cells at each program point. The analysis is based on the
principles of abstract interpretation, specifically using an interval based
abstract domain. This report documents the design choices and implementation
details of the analysis. The project is based upon the already implemented
ebpf-tools\cite{ebpf}. Ebpf-tools is able to generate control-flow graphs
from the micro-ebpf language. These control-flow graphs are the foundation of
the implementation provided within this project.

\subsection*{Abstract domain}
Throughout this project, I have used the infinite intervals to represent the
possible values of both registers and memory locations. That is to say, that an
abstract value is an interval $[l, u]$ where $l, u \in \{-\infty, \infty\}$.
These infinite intervals require the use of widening, and subsequent narrowing
to give useable results.

In the micro-ebpf language we have 11 registers (r0-r10) and a memory section
Mem, which contains 512 different cells. Throughout the implementation both
registers and memory cells were treated in the same manner. That is to say,
that a register/memory cell could take on any value in the interval mentioned
previously, or they could be some unreachable state denoted by $\bot$. Also,
the interval $\{-\infty, \infty\}$ itself is the representation of $\top$,
indicating that the register/memory cell can take on any value.

\subsection*{Modelling program states}
The modelling of program states are best described by an example. Say we have
some arbitrary node, n, within the control flow graph. Now given some
transition $t$ from $n \to n'$ through some expression $exp$, then the state at
$n'$, namely $s'$, is given by:

\begin{align}
  s' &= (\llbracket exp \rrbracket s) \sqcup_{Itv} t'
\end{align}

Where $t'$ is any other transition, which also ends in n'. That is to say, that
the new state s' at node n' is given by the union of all the states that came
before and end in the current node. For this to work probably we first
initialize all states to the neutral element of the interval union, which is
$\bot$.


\subsection*{Abstract interpretation of expressions}
In general my implementation follows the abstract interpretation of expressions
seen in mine\cite{mine}. However, mine does not provide a definition for the
binary operators. In my case I was not able to get the time to implement the
binary operators. As such, if we ever see a binary operator, then we
immediately just return $\top$.

\subsubsection*{Memory lookups}
For a memory lookup \texttt{Mem[ri]}, the abstract value of the register
\texttt{ri} determines the possible memory addresses to be read. Let $[l, u]$
denote the abstract interval of \texttt{ri}. The set of possible addresses is
$\{l, l+1, \dots, u\}$, meaning the result of the memory lookup is the union of
the abstract values at these addresses:
\begin{align}
  \text{Let } R(s)&= \bigsqcup_{Itv_{i \in [l, u]}} \llbracket \texttt{r[i]} \rrbracket s \\
  \llbracket \texttt{Mem[ri]} \rrbracket s &= \bigsqcup_{Itv_{j \in R(s)}} \texttt{Mem[j]}
\end{align}

This means, that there could be cases where we are writing to multiple memory
cells at the same time, if we are using a register, which at the current state
is not a constant value.


\subsection*{Iteration strategy}
In this project the iteration strategy used is the work-set algorithm. This
method was used to avoid re-evaluating parts of the program, that were not
affected by the previous state. For more details on the specific implementation
see Section~\ref{sec:workset}.

The work-set algorithm was implemented with widening and narrowing. Widening
($\nabla$) is defined as in Mine\cite{mine}, i.e. we have that: 
\begin{align*}
  [l_1, u_1] \nabla [l_2, u_2] = [&\text{if } l_2 < l_1 \text{
  then } - \infty \text{ else } l_1, \\ 
  &\text{if } u_2 > u_1 \text{ then }
  \infty \text{ else } u_1]
\end{align*}

And Narrowing ($\Delta$) is defined as:
\begin{align*}
  [l_1, u_1] \Delta [l_2, u_2] = [&\text{if } l_1 = -\infty \text{
  then } l_2 \text{ else } l_1, \\ 
  &\text{if } u_1 = \infty \text{ then }
  u_2 \text{ else } u_1]
\end{align*}

To not overdo this, I included a counter for each node in the control flow
graph. If we reach the node x times, then we perform one widening step.
Afterwards, we perform some y amount of narrowing steps, but to avoid
infinitely looping, we limit y to some finite number.



\section{Implementation}

\subsection{Definitions}\label{sec:def}
\begin{lstlisting}[language={haskell}, caption={Bound data type}, label={lst:def}]
data Bound = NegInf | Val Integer | PosInf deriving (Eq, Ord)
data Interval = Interval Bound Bound deriving (Eq, Ord)

// Bottom M
data BottomM a = Bottom | Value a
  deriving (Eq, Ord)
\end{lstlisting}
As mentioned I make use of the infinite interval. This is of course not
something machines can correctly represent, and as such I use the Bound data
type, which's definition can be seen in Listing~\ref{lst:def}. Bound can thus
take up 3 different "values". Either it is NegInf($-\infty$), some value or
PosInf($\infty$). Note, that because we define it in the order we have, then
NegInf will be seen as smaller than any Val and PosInf will be seen as larger
than any Val. This allows us to simply use the default min and max functions on
the Bound type.

Along with Bound there is also the definition of the Interval type, which
contains two Bound values, the first indicating the lower bound and the second
indicating the upper bound.

This now allows me to represent all the interval values except for $\bot$. For
$\bot$ instead of simply adding the value to the Interval type, I decided to
make use of BottomM. BottomM is defined exactly the same way as the Maybe monad
usually used in Haskell. The only difference between the two is that BottomM
makes use of Bottom instead of Nothing, and Value x instead of Just x.

This choice was made as the large majority of operations on both intervals and
bounds return Bottom, if any of the inputs are Bottom. This is the exact same
functionality from the Maybe monad, and saves a lot of repetitive code checking
the specific cases.

\subsection{Operator implementation}
The definitions from Section~\ref{sec:def} allows me to define functions as the type signature:
\begin{lstlisting}[language={haskell}]
binaryBound :: Bound -> Bound -> BottomM Bound
unaryBound :: Bound -> BottomM Bound

// Interval operators
type IntervalM = BottomM Interval
binaryInterval ::  Interval -> Interval -> IntervalM
unaryInterval ::  Interval -> Interval -> IntervalM

// Interval comparisons
binaryInterval :: Interval -> Interval -> 
                  BottomM (Interval, Interval)
unaryInterval ::  Interval -> 
                  BottomM (Interval, Interval)
\end{lstlisting}

\subsubsection*{Examples}
\begin{lstlisting}[language={haskell}, caption={Full division implementation}, label={lst:div}]
divBound :: Bound -> Bound -> BottomM Bound
divBound (Val x) (Val y)
  | y == 0 = Bottom
  | otherwise = Value $ Val (x `div` y)
divBound _ PosInf = Value $ Val 0
divBound _ NegInf = Value $ Val 0
divBound NegInf (Val y)
  | y > 0 = Value NegInf
  | y < 0 = Value PosInf
  | y == 0 = Bottom
divBound PosInf (Val y)
  | y > 0 = Value PosInf
  | y < 0 = Value NegInf
  | y == 0 = Bottom
divBound _ _ = Bottom

divInterval :: Interval -> Interval -> IntervalM
divInterval ab@(Interval a b) cd@(Interval c d)
  | Val 1 <= c = do
      ac <- divBound a c
      ad <- divBound a d
      bc <- divBound b c
      bd <- divBound b d
      Value $ Interval (min ac ad) (max bc bd)
  | d <= Val (-1) = do
      bc <- divBound b c
      bd <- divBound b d
      ac <- divBound a c
      ad <- divBound a d
      Value $ Interval (min bc bd) (max ac ad)
  | otherwise =
      let first = do
            t1 <- intersectInterval cd (Interval (Val 1) PosInf)
            divInterval ab t1
          second = do
            t2 <- intersectInterval 
                    cd
                    (Interval NegInf (Val (-1)))
            divInterval ab t2
       in unionIntervalM first second
\end{lstlisting}  
\paragraph{Division:}
As an example of division of intervals are provided in Listing~\ref{lst:div}.
divBound follows our general intuition, where the main difference is that
division by 0 is not undefined, but rather just returns bottom. 

The divInterval is more interesting. It first makes sure that c is greater than
or equal to 1. That way we are not dividing by 0, which would result in bottom,
which we do not want for the entire interval. If the check for c fails, then we
instead attempt to evaluate if d is less than or equal to -1. This would be the
case, if the entire interval is negative. If both of these evaluations fails,
then we know that we do not have an entirely positive nor entirely negative
interval, and as such we go to the recursive call. 

In the recursive call, we simply find the intersection between all possible
positive numbers and our divisor, and attempt to divide ab with it and then the
same for the negative values. At last, we just take the union between the two
intervals.

In this case, there is never the possibility of dividing by zero, but using the
bottomM monad, we don't have to check for this case explicitly.


\begin{lstlisting}[language={haskell}, caption={Union interval implementation}, label={lst:union}]
unionIntervalM :: IntervalM -> IntervalM -> IntervalM
unionIntervalM m1 m2 =
  case (m1, m2) of
    (Bottom, Value i2) -> Value i2
    (Value i1, Bottom) -> Value i1
    (Value i1, Value i2) -> unionInterval i1 i2
    (_, _) -> Bottom
\end{lstlisting}
\paragraph{Union Interval:} The union interval is the one operator for which Bottom is the neutral element.
This means, that if one of the inputs is Bottom, then we should not just return
Bottom ourselves. As such, the implementation of union can be seen in
Listing~\ref{lst:union}, where if only one input is bottom, then we simply
return the other input.



\subsection{Work-set algorithm}\label{sec:workset}

\section{Validation}
Validation of the implementation includes both unit tests\footnote{These
unit tests were created with the help of LLMs, where I then looked over to make
sure the test results made sense.} and the default programs provided by
eBPF-tools, where the output was simply checked by hand. Some programs with
load and store instructions were also created and checked by hand to make sure
that the implementation gave expected results.

\section{Future work}
Although the implementation seems to generally work, there are some areas,
which I simply have not had the time to improve upon. These parts are
referenced here as what could be worked on further.

\subsection{Optimizations}
By no means is this implementation optimal. The general goal of the project is
to keep it as easy to understand as possible, and this has a large effect on
the overall optimizations. There are multiple cases of this, but I will just
mention what I think would lead to the best performance gain:

\begin{itemize}
  \item Making it such that we don't store the state of every single node. The
    current implementation stores the state for every node in the CFG. This is
    not efficient at all. If a node is only visited once, then we could pass
    the state it produces directly on to the next node, and then never have to
    save the state of the incoming node. We would most likely have to do one
    pass through of the CFG to check for such nodes, but afterwards we would
    save a lot of the memory usage that probably occur on larger programs.
  \item Narrowing and Widening are currently implemented by checking every
    single register and memory cell to see if the cell/register has changed
    since the last state. This is highly inefficient, and it should be possible
    to reduce the number of checked elements. For example for an instruction
    which sends the value to a register, then we would only have to check that
    single register. The reason this was not implemented is because of instructions such as:
    \begin{verbatim}
      stb [r0], 10
    \end{verbatim}
    Here we are writing to the memory location by the value of r0. r0 is seen
    as an interval, so in this case we would have to handle the interval as the
    possible memory cells that have to be updated instead of a single index.
    For the sake of time, this was not implemented
\end{itemize}


\subsection{Unhandled actions}
There are two main areas where I have taken a shortcut and not handled the instructions. Those two are explained here.

\paragraph{Binary operators: } The binary operators are not evaluated, instead
I simply return the top interval whenever we reach a binary operator. Binary
operators would of course give us some further information about the current
interval, but due to time constraints these were not implemented.

\paragraph{Conditional jumps: } 
In the provided snippet is how we handle the conditional jumps.
\begin{lstlisting}[language={haskell}, numbers=none]
case jmp of
  Jeq -> equalInterval i1 i2
  Jne -> notEqualInterval i1 i2
  Jlt -> lessThanInterval i1 i2
  Jle -> lessThanEqualInterval i1 i2
  Jgt -> greaterThanInterval i1 i2
  Jge -> greaterThanEqualInterval i1 i2
  Jslt -> lessThanInterval i1 i2
  Jsle -> lessThanEqualInterval i1 i2
  Jsgt -> greaterThanInterval i1 i2
  Jsge -> greaterThanEqualInterval i1 i2
  _ -> Value (i1, i2) -- For other jumps, we don't refine
\end{lstlisting}

Here we can see, that we do not take the information about the signed and
unsigned use into account, and we do also not match for the Jset instruction.
The Jset instruction is simply not implemented, and we just return our input
and with the signed and unsigned one could argue, that we should be more
restrictive on the interval. If we ever reach an unsigned instruction, then we
would know that the register could never be negative, instead of just using the
default interval calculations.

\section{Conclusion}
This report has outlined the design of a static value analysis for a subset of eBPF. The analysis uses abstract interpretation with an interval domain to approximate the values of registers and memory. We have discussed the abstract domain, state representation, abstract transformers, and the iterative algorithm to find a fixpoint. The use of widening is crucial for guaranteeing termination of the analysis in the presence of loops.



\bibliographystyle{plain}   % or try alpha, unsrt, abbrv
\bibliography{refs}         % refs.bib file

\end{document}
